\documentclass{article}

% 这是一段注释

% 导言区

\usepackage{array}
\usepackage{xeCJK}
\usepackage{amsmath}

\title{ZYZ EXTRA Contest \#233}
\author{ZYZOI}
\date{May 30th, 2020}

\begin{document}
  % 排版标题、作者、日期 
  \maketitle

  \begin{center}
  \textbf{(请选手务必仔细阅读本页内容)}
  \end{center}

  % generate from https://www.tablesgenerator.com/latex_tables

  \begin{center}
  % \begin{tabular}{*{4}{|C{9em}}|}
  \begin{tabular}{|p{3cm}|p{3cm}|p{3cm}|p{3cm}|}
    \hline
    题目名称   & A + B Problem & A - B Problem & A * B Problem \\ \hline
    可执行文件名 & aplsb         & amnsb         & amulb         \\ \hline
    源程序文件名 & aplsb.cpp     & amnsb.cpp     & amulb.cpp     \\ \hline
    输入文件名  & aplsb.in      & amnsb.in      & amulb.in      \\ \hline
    输出文件名  & aplsb.out     & amnsb.out     & amulb.out     \\ \hline
    时间限制   & 1.0 s         & 1.0 s         & 1.0 s         \\ \hline
    空间限制   & 128 MiB       & 128 MiB       & 128 MiB       \\ \hline
    测试点数目  & 10            & 10            & 10            \\ \hline
    测试点分值  & 10            & 10            & 10            \\ \hline
    是否有部分分 & 否             & 否             & 否             \\ \hline
    题目类型   & 传统            & 传统            & 传统            \\ \hline
  \end{tabular}
  \end{center}

  \begin{center}
    编译开关 \\

    \begin{tabular}{|p{3cm}|p{3cm}|p{3cm}|p{3cm}|}
    % \begin{tabular}{|c{3cm}|c{3cm}|c{3cm}|c{3cm}|}
      \hline
      对于 C++ 语言 & -lm & -lm & -lm \\ \hline
    \end{tabular}
  \end{center}

  \newpage

  \section{A + B Problem}

  \subsection{题目描述}

  输入 $ a $ 和 $ b $,输出 $ a + b $ 的结果.

  \subsection{输入格式}

  一行两个正整数 $ a $ 和 $ b $.

  \subsection{输出格式}

  一行一个正整数 $ a + b $.

  \subsection{样例}

  \subsubsection{样例输入 1}

  1 2

  \subsubsection{样例输出 1}

  3
  
  \subsubsection{样例解释 1}

  注意到 $a + b = 1 + 2 = 3$, 因而最后的答案为 $3$.

  \subsection{数据范围与约定}

  对于 $ 100\% $ 的数据, $ 1 \leq a, b \leq 10 ^ 6 $.

  \newpage

  \section{A - B Problem}

  \subsection{题目描述}

  \subsection{输入格式}

  \subsection{输出格式}

  \subsection{样例}

  \subsubsection{样例输入 1}

  \subsubsection{样例输出 1}

  \subsection{数据范围}

  \newpage

  \section{A * B Problem}

  \subsection{题目描述}

  \subsection{输入格式}

  \subsection{输出格式}

  \subsection{样例}

  \subsubsection{样例输入 1}

  \subsubsection{样例输出 1}

  \subsection{数据范围}

\end{document}
